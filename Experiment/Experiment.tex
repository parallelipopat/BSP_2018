\documentclass[../Main.tex]{subfiles}

\begin{document}

\section{Experiment}
We now look at running the simulation, and analyzing the results. First, we look at some theoretical results pertaining to the four geometrical invariants under question - time reversibilty, total energy, total linear momentum, and total angular momentum. Then, we look at running the simulation for different values of the parameters $\beta$ and $\gamma$.
We will assume that the simulations have the following initial configuration (unless stated otherwise):

\begin{table}[H]
	\centering
	\begin{tabular}{ |l|r| }
		\hline
		Parameter & Initial Value \\
		\hline
		No. of particles & 864 \\
		Temperature & 1 K\textsuperscript{*} \\
		Density & $38.744 \mbox{ m}^{*}\sigma^{*^{-3}}$ \\
		h & 0.032 \\
		$N_{e}$ & 100 \\
		$N_{f}$ & 500 \\
		$N_{s}$ & 10 \\
		$N_{n}$ & 15 \\
		\hline
	\end{tabular}
	\caption{Initial Configuration for molecular simulations}
	\label{tbl:initial_configuration_simulation}
\end{table}

The last four elements in the table indicate that the simulation will initially be equilibrated for 100 iterations, and then run for 500 iterations where samples will be collected every 10 iterations (giving a total of 50 samples). The list of neighbours will be updated every 10 iterations in both the equilibration and final stages.

\subsection{Geometric Invariants}

We first expand on the geometric invariants of the molecular dynamics system. \\\
The system is time reversible - if we go from state $s_{1}$ to state $s_{2}$ in time $\delta t$, then we return to state $s_{1}$ from state $s_{2}$ in time $\delta t$ by reversing the signs on the velocities. This can be seen by applying the transformation $t \mapsto -t$ to Equation~\ref{eqn:mol_hamiltonian_separate} - the position $q$ does not change, but the sign on momentum $p$ gets reversed.
\begin{align*}
q \mapsto q = \tilde{q},  \quad \quad p \mapsto -p = \tilde{p}.
\end{align*}
Then
\begin{alignat*}{4}
	\dot{\tilde{q}} &= \frac{d\tilde{q}}{d\left(-t\right)} \quad &&= -\nabla_{\tilde{p}}H \quad &&= \nabla_{p}H \quad &&= \dot{q},\\
	\dot{\tilde{p}} &= \frac{d\tilde{p}}{d\left(-t\right)} \quad &&= -\nabla_{\tilde{q}}H \quad && =-\nabla_{q}H \quad &&= \dot{p}.
\end{alignat*}
Thus, the Hamiltonian system is time reversible.

Conservation of total energy, linear momentum and angular momentum follow readily from Newton's Laws of Motion, as there is no external force on the system. We introduce the concept of first integrals that will make it simpler to check which methods conserve energy and momentum. \\
A non-constant function $I(y)$ is a \textit{first integral} (or invariant) of the  differential equation $\dot{y} = F(y)$ if $I(y(t))$ is constant along every solution, or equivalently, if
\begin{align}
\nabla I(y)F(y) = 0    \quad \quad \quad  \forall y. \quad \cite{HarierLubichWanner2003}
\end{align}
We now give proofs that the total energy, total linear momentum, and total angular momentum are first integrals \cite{HarierLubichWanner2003}.
\begin{theorem} The total energy, or the Hamiltonian (Equation~\ref{eqn:mol_hamiltonian}) is a first integral. \end{theorem}
\begin{proof}
$$
H(p, q) = \frac{1}{2}\sum_{i = 1}^{N} m_{i}^{-1}p_{i}^{T}p_{i} + \sum_{i=1}^{N-1} \sum_{j=i+1}^{N} V_{ij}\left(\left\Vert q_{i} - q_{j} \right\Vert\right)
$$
So,
$$
	H' = \left(\nabla_{p}H^{T}, \nabla_{q}H^{T}\right)
$$
and
$$
	\frac{dH}{dt} =
		\begin{pmatrix*}[r]
			-\nabla_{q}H \\
			\nabla_{p}H
		\end{pmatrix*}
$$
Then, 
$$
H'(p, q) \cdot \frac{dH}{dt} = \nabla_{p}H^{T}\cdot-\nabla_{q}H + \nabla_{q}H^{T}\cdot \nabla_{p}H = 0
$$ 
\end{proof} 

\begin{theorem} Total linear momentum $P = \sum_{i=1}^{N} p_{i}$ is a first integral. \end{theorem}
\begin{proof}
\begin{align*}
\nabla P & = \frac{dP}{dt} = \sum_{i=1}^{N} \frac{dp_{i}}{dt} \\
&=\sum_{i=1}^{N}\sum_{j=1}^{N}\nu_{ij}\left(q_{i} - q_{j}\right) \\
& = 0 \qquad \qquad \qquad \qquad \cdots \mbox{ as }\nu_{ij} = \nu_{ji} \forall i, j
\end{align*}
\end{proof} 

\begin{theorem} Total angular momentum $L = \sum_{i=1}^{N} q_{i} \times p_{i}$ is a first integral. \end{theorem}
\begin{proof}
\begin{align*}
\nabla L & = \sum_{i=1}^{N} \frac{d}{dt} \left(q_{i} \times p_{i}\right) \\
& = \sum_{i=1}^{N} \left(\dot{q}_{i} \times p_{i} + {q}_{i} \times \dot{p}_{i}\right) \\  
& = \sum_{i=1}^{N} \left( \frac{1}{m_{i}}p_{i} \times p_{i} + q_{i} \times \sum_{j=1}^{N}\nu_{ij}\left(q_{i} - q_{j}\right)\right) \\  
& = \sum_{i=1}^{N} \left( \frac{1}{m_{i}}p_{i} \times p_{i}\right) +  \sum_{i=1}^{N}\sum_{j=1}^{N} \left(q_{i} \times \nu_{ij}\left(q_{i} - q_{j}\right)\right) \\
& = 0  \qquad \qquad \qquad \qquad \cdots \mbox{ as }\nu_{ij} = \nu_{ji} \mbox{ and } p_{i} \times p_{i} =  \forall i, j
\end{align*}
\end{proof}
 We have that total energy and total angular momentum are quadratic first integrals, and total linear momentum - as the name suggests - is a linear first integral.

Most numerical methods preserve linear first integrals - in fact, it is a property shared by all Runge-Kutta methods. Indeed, this holds true for the entire family of Newmark Beta methods as well.

\begin{claim} All Newmark Beta methods preserve linear first integrals \end{claim}
\begin{proof}
Let $I(\vec{x}, \vec{v}) = b^{T}\vec{x} + c^{T}\vec{v}$ be a linear first integral, where $b$ and $c$ are some constant vectors.
By the defintion of first integrals, we want that $I'(\vec{x}, \vec{v})= 0$ for all $\vec{x}, \vec{v}$.
But $I'(\vec{x}, \vec{v}) = b^{T}\vec{v} + c^{T}\vec{a(\vec{x})}\ = 0$, implying that $b^{T} = 0$ and $c^{T}\vec{a} = 0$ for all $\vec{x}$. So, $I(\vec{x}, \vec{v}) = I(\vec{v}) = c^{T}\vec{v}$ (This makes sense as a linear first integral, total linear momentum, depends only on the velocities of the particles, and not the positions).\\
Now, we multiply $c^{T}$ to the velocity portion of the Newmark Beta method and get that
\begin{align*}
I\left(\vec{v}_{i+1}\right) &= c^{T}\vec{v}_{i+1} \\
&= c^{T}\vec{v}_{i} + h\left[\left(1-\gamma\right)c^{T}\vec{a}(\vec{x}_{i}) + \gamma c^{T}\vec{a}(\vec{x}_{i+1})\right] \\
&= c^{T}\vec{v}_{i} = I\left(\vec{v}_{i}\right)
\end{align*} 
\end{proof}

We cannot make a similar claim regarding the preservation of quadratic first integrals, but we can certainly simplify our work. Noether's Theorem states for the Hamiltonian $H(p, q)$ of the form Equation~\ref{eqn:mol_hamiltonian}, all quad

\subsection{Explicit Newmark Beta schemes}

Recall from Equation~\ref{eqn:newmark-beta} that a Newmark Beta method has the form
\begin{align*}
		\vec{v}_{i+1} & = \vec{v}_{i} + h\left[\left(1-\gamma\right)\vec{a}_{i} + \gamma\vec{a}_{i+1}\left(\vec{x}_{i+1}\right)\right], \\
		\vec{x}_{i+1} & = \vec{x}_{i} + h\vec{v}_{i} + \frac{h^2}{2}\left[\left(1-2\beta \right)\vec{a}_{i} + 2\beta\vec{a}_{i+1}\left(\vec{x}_{i+1}\right)\right], 
\end{align*}with $\beta \in \left[0,0.5\right]$ and $\gamma \in \left[0,1\right]$.
Setting $\beta = 0$, we get
\begin{align}
	\begin{split}
		\vec{v}_{i+1} & = \vec{v}_{i} + h\left[\left(1-\gamma\right)\vec{a}_{i} + \gamma\vec{a}_{i+1}\right], \\
		\vec{x}_{i+1} & = \vec{x}_{i} + h\vec{v}_{i} + \frac{h^2}{2}\vec{a}_{i},
	\end{split} \label{explicit_newmark-beta}  
\end{align}
which is explicit in both equations (and hence, known as the Explicit Newmark Beta form).

\subsubsection{Preservation of Reversibility}
It is pretty straightforward to find the values of $\gamma$ that preserve the reversibility to the system. From \cite{Paganini2017}, we have
\begin{theorem}The maximal order of a reversible one-step method is always even. \label{thm:reversible_even_order} \end{theorem}
 As Newmark-Beta methods have maximal order 2 only if $\gamma = \frac{1}{2}$, we get from the contrapositive of \ref{thm:reversible_even_order} that any explicit scheme with $\gamma \neq \frac{1}{2}$ is not reversible.
\begin{claim} The Velocity Verlet algorithm is reversible. \end{claim}
\begin{proof}
The Velocity Verlet algorithm is
\begin{align}
		\vec{v}_{i+1} & = \vec{v}_{i} + h\left(\frac{1}{2}\vec{a}_{i} + \frac{1}{2}\vec{a}_{i+1}\right), \label{eqn:target_reversible_velocity_verlet_v}\\
		\vec{x}_{i+1} & = \vec{x}_{i} + h\vec{v}_{i} + \frac{h^2}{2}\vec{a}_{i} \label{eqn:target_reversible_velocity_verlet_x} 
\end{align}
Swapping $i \leftrightarrow i+1$ and $h \leftrightarrow -h$
\begin{align}
		\vec{v}_{i} & = \vec{v}_{i+1} - h\left(\frac{1}{2}\vec{a}_{i+1} + \frac{1}{2}\vec{a}_{i}\right),  \label{eqn:current_reversible_velocity_verlet_v}\\
		\vec{x}_{i} & = \vec{x}_{i+1} - h\vec{v}_{i+1} + \frac{h^2}{2}\vec{a}_{i+1} \label{eqn:current_reversible_velocity_verlet_x} 
\end{align}
It is straight-forward to see that Equation~\ref{eqn:current_reversible_velocity_verlet_v} is the same as Equation~\ref{eqn:target_reversible_velocity_verlet_v}.
For Equation~\ref{eqn:current_reversible_velocity_verlet_x}, we see
\begin{align*}
	\vec{x}_{i+1} & = \vec{x}_{i} + h\vec{v}_{i+1} - \frac{h^2}{2}\vec{a}_{i+1} \\
	& = \vec{x}_{i} + h\left(\vec{v}_{i} + \frac{h}{2}\vec{a}_{i} + \frac{h}{2}\vec{a}_{i+1}\right) - \frac{h^2}{2}\vec{a}_{i+1} \\
	& = \vec{x}_{i} + h\vec{v}_{i} + \frac{h^2}{2}\vec{a}_{i}
\end{align*} 
\end{proof} 


\subsection{Implicit Newmark Beta schemes}

\end{document}
