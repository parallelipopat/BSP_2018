\documentclass[../Main.tex]{subfiles}

\begin{document}

\section{Experiment}
We now look at running the simulation, and analyzing the results. First, we look at some theoretical results pertaining to the four geometrical invariants under question - time reversibilty, total energy, total linear momentum, and total angular momentum. Then, we look at running the simulation for different values of the parameters $\beta$ and $\gamma$.
We will assume that the simulations have the following initial configuration (unless stated otherwise):

\begin{table}[H]
	\centering
	\begin{tabular}{ |l|r| }
		\hline
		Parameter & Initial Value \\
		\hline
		No. of particles & 864 \\
		Temperature & 1 K\textsuperscript{*} \\
		Density & $38.744 \mbox{ m}^{*}\sigma^{*^{-3}}$ \\
		h & 0.032 \\
		$N_{e}$ & 100 \\
		$N_{f}$ & 500 \\
		$N_{s}$ & 10 \\
		$N_{n}$ & 15 \\
		\hline
	\end{tabular}
	\caption{Initial Configuration for molecular simulations}
	\label{tbl:initial_configuration_simulation}
\end{table}

The last four elements in the table indicate that the simulation will initially be equilibrated for 100 iterations, and then run for 500 iterations where samples will be collected every 10 iterations (giving a total of 50 samples). The list of neighbours will be updated every 10 iterations in both the equilibration and final stages.

\subsection{Time Reversibility}
The molecular dynamics system is time reversible - if we go from state $s_{1}$ to state $s_{2}$ in time $\delta t$, then we can return to state $s_{1}$ from state $s_{2}$ in time $\delta t$ by reversing the signs on the velocities. This can be seen formally by applying the transformation $t \mapsto -t$ to Equation~\ref{eqn:mol_hamiltonian_separate} - the position $q$ does not change, but the sign on momentum $p$ gets reversed.
\begin{align*}
q \mapsto q = \tilde{q},  \quad \quad p \mapsto -p = \tilde{p}.
\end{align*}
Then
\begin{alignat*}{4}
	\dot{\tilde{q}} &= \frac{d\tilde{q}}{d\left(-t\right)} \quad &&= -\nabla_{\tilde{p}}H \quad &&= \nabla_{p}H \quad &&= \dot{q},\\
	\dot{\tilde{p}} &= \frac{d\tilde{p}}{d\left(-t\right)} \quad &&= -\nabla_{\tilde{q}}H \quad && =-\nabla_{q}H \quad &&= \dot{p}.
\end{alignat*}
Thus, the Hamiltonian system is time reversible.

We say that a numerical one-step method $\Phi_{h}: (\vec{x}_{i},\vec{v}_{i}) \mapsto (\vec{x}_{i+1},\vec{v}_{i+1})$ is time reversible if $\Phi_{h} = \Phi_{-h}^{-1}$. Informally, this means that if we exchange $i \leftrightarrow i+1$ and replace $h$ by $-h$ in our origina method, then we should get the same method back.
In order to ascertain which Newmark Beta methods are reversible, we cite \cite{Paganini2017}:
\begin{theorem}The maximal order of a reversible one-step method is always even. \label{thm:reversible_even_order} \end{theorem}
 As Newmark Beta methods have maximal order 2 only if $\gamma = \frac{1}{2}$, the contrapositive of \ref{thm:reversible_even_order} states that any scheme with $\gamma \neq \frac{1}{2}$ is not reversible. Now, we claim
\begin{claim} Any Newmark Beta method with $\gamma = \frac{1}{2}$  is reversible. \end{claim}
\begin{proof}
A Newmark Beta scheme with $\gamma = \frac{1}{2}$ has the form
\begin{align}
		\vec{v}_{i+1} & = \vec{v}_{i} + h\left(\frac{1}{2}\vec{a}_{i} + \frac{1}{2}\vec{a}_{i+1}\right), \label{eqn:target_reversible_newmark-beta_v}\\
		\vec{x}_{i+1} & = \vec{x}_{i} + h\vec{v}_{i} + \frac{h^2}{2}\left[(1-2\beta)\vec{a}_{i} + 2\beta\vec{a}_{i+1}\right].  \label{eqn:target_reversible_newmark-beta_x} 
\end{align}
Swapping $i \leftrightarrow i+1$ and replacing $h$ with $-h$
\begin{align}
		\vec{v}_{i} & = \vec{v}_{i+1} - h\left(\frac{1}{2}\vec{a}_{i+1} + \frac{1}{2}\vec{a}_{i}\right),  \label{eqn:current_reversible_newmark-beta_v}\\
		\vec{x}_{i} & = \vec{x}_{i+1} - h\vec{v}_{i+1} + \frac{h^2}{2}\left[(1-2\beta)\vec{a}_{i+1} + 2\beta\vec{a}_{i}\right]. \label{eqn:current_reversible_newmark-beta_x} 
\end{align}
It is straight-forward to see that Equation~\ref{eqn:current_reversible_newmark-beta_v} is the same as Equation~\ref{eqn:target_reversible_newmark-beta_v}.
For Equation~\ref{eqn:current_reversible_newmark-beta_x}, we see
\begin{align*}
	\vec{x}_{i+1} & = \vec{x}_{i} + h\vec{v}_{i+1} - \frac{h^2}{2}\left[(1-2\beta)\vec{a}_{i+1} + 2\beta\vec{a}_{i}\right] \\
	& = \vec{x}_{i} + h\left(\vec{v}_{i} + \frac{h}{2}\vec{a}_{i} + \frac{h}{2}\vec{a}_{i+1}\right) - \frac{h^2}{2}\left[(1-2\beta)\vec{a}_{i+1} + 2\beta\vec{a}_{i}\right] \\
	& = \vec{x}_{i} + h\vec{v}_{i} + \frac{h^2}{2}\left[\left(1-2\beta\right)\vec{a}_{i} + 2\beta\vec{a}_{i+1}\right].
\end{align*} 
\end{proof} 
 
While the schemes may be time reversible in theory, the rounding-off errors in practice mean that we may not get back the initial state of the system after reversing the velocities. We test this for different values of $\beta$ and $\gamma$ on a short-term, and medium-term scale, and on a long-term scale for the explicit Newmark Beta methods (due to much faster execution time). In these experiments, we do not differentiate between equilibration and final iterations, and reverse the sign on the velocities after half of the total number of iterations have been completed.



\subsection*{Aside: First Integrals}
The next three invariants - total energy, linear momentum, and angular momentum - follow readily from Newton's Laws of Motion as there is no external force on the system. In order to simplify checking which methods preserve these invariants the best, we introduce the concept of first integrals. \\
A non-constant function $I(y)$ is a \textit{first integral} (or invariant) of the differential equation $\dot{y} = F(y)$ if $I(y(t))$ is constant along every solution, or equivalently, if
\begin{align}
\nabla I(y)F(y) = 0    \quad \quad \quad  \forall y. \quad \cite{HarierLubichWanner2003} \label{eqn:grad_first_integral_zero}
\end{align}
Total energy, total linear momentum, and total angular momentum are all first integrals \cite{HarierLubichWanner2003}. We look at proving this in the next subsections, adapting the proofs from \cite{HarierLubichWanner2003}.

\subsection{Total Linear Momentum}
 
\begin{theorem} Total linear momentum $P = \sum_{i=1}^{N} p_{i}$ is a first integral. \end{theorem}
\begin{proof}
\begin{align*}
\nabla P & = \frac{dP}{dt} = \sum_{i=1}^{N} \frac{dp_{i}}{dt} \\
&=\sum_{i=1}^{N}\sum_{j=1}^{N}\nu_{ij}\left(q_{i} - q_{j}\right) \\
& = 0 \qquad \qquad \qquad \qquad \cdots \mbox{ as }\nu_{ij} = \nu_{ji} \forall i, j
\end{align*}
\end{proof} 

Total linear momentum - as the name suggests - is a linear first integral. Most numerical methods preserve linear first integrals - in fact, it is a property shared by all Runge-Kutta methods. Indeed, this holds true for the entire family of Newmark Beta methods as well.
\begin{claim} All Newmark Beta methods preserve linear first integrals \end{claim}
\begin{proof}
Let $I(\vec{x}, \vec{v}) = b^{T}\vec{x} + c^{T}\vec{v}$ be a linear first integral, where $b$ and $c$ are some constant vectors.
By the defintion of first integrals, we want that $I'(\vec{x}, \vec{v})= 0$ for all $\vec{x}, \vec{v}$.
But $I'(\vec{x}, \vec{v}) = b^{T}\vec{v} + c^{T}\vec{a(\vec{x})}\ = 0$, implying that $b^{T} = 0$ and $c^{T}\vec{a}(\vec{x}) = 0$ for all $\vec{x}$. So, $I(\vec{x}, \vec{v}) = I(\vec{v}) = c^{T}\vec{v}$ (This makes sense in a real-world scenario as total linear momentum depends only on the velocity of the body, and not its position).\\
Now, we multiply $c^{T}$ to the velocity portion of the Newmark Beta method and get that
\begin{align*}
I\left(\vec{v}_{i+1}\right) &= c^{T}\vec{v}_{i+1} \\
&= c^{T}\vec{v}_{i} + h\left[\left(1-\gamma\right)c^{T}\vec{a}(\vec{x}_{i}) + \gamma c^{T}\vec{a}(\vec{x}_{i+1})\right] \\
&= c^{T}\vec{v}_{i} = I\left(\vec{v}_{i}\right)
\end{align*} 
\end{proof}

As we calibrated the system to have an initial total linear momentum of zero, the total linear momentum will be zero for all of the trials. As this holds true for all members of the Newmar Beta family, we do not present experimental results for this property, but rather state that in all trials, the total linear momentum was almost zero - it cannot be exactly zero due to inaccuracies introduced by floating-point arithmetic and rounding-off, but these errors were in the range of $10^-{10}$.

\subsection{Total Angular Momentum}

\begin{theorem} Total angular momentum $L = \sum_{i=1}^{N} q_{i} \times p_{i}$ is a first integral. \end{theorem}
\begin{proof}
\begin{align*}
\nabla L & = \sum_{i=1}^{N} \frac{d}{dt} \left(q_{i} \times p_{i}\right) \\
& = \sum_{i=1}^{N} \left(\dot{q}_{i} \times p_{i} + {q}_{i} \times \dot{p}_{i}\right) \\  
& = \sum_{i=1}^{N} \left( \frac{1}{m_{i}}p_{i} \times p_{i} + q_{i} \times \sum_{j=1}^{N}\nu_{ij}\left(q_{i} - q_{j}\right)\right) \\  
& = \sum_{i=1}^{N} \left( \frac{1}{m_{i}}p_{i} \times p_{i}\right) +  \sum_{i=1}^{N}\sum_{j=1}^{N} \left(q_{i} \times \nu_{ij}\left(q_{i} - q_{j}\right)\right) \\
& = 0  \qquad \qquad \qquad \qquad \cdots \mbox{ as }\nu_{ij} = \nu_{ji} \mbox{ and } p_{i} \times p_{i} = 0 \,\forall i, j
\end{align*}
\end{proof}

Total angular momentum is a quadratic first integral. By Noether's Theorem, it has the form $I(\vec{x}, \vec{v}) = \vec{v}^{T}\left(C\vec{x} + d\right)$, where $C$ is a constant square matrix and $d$ is a constant vector \cite{HarierLubichWanner2003}.
We have the Velocity Verlet method preserves quadratic integrals of this form, and hence, total angular momentum. To see this, we need to express it as the composition of two half-step methods:
\begin{align}
	\begin{split}
		\vec{v}_{i+\frac{1}{2}} = \vec{v}_{i} + \frac{h}{2}\vec{a}(\vec{x}_{i}) \\
		\vec{x}_{i+\frac{1}{2}} = \vec{x}_{i} + \frac{h}{2}\vec{v}_{i+\frac{1}{2}}
	\end{split}
\end{align}
and
\begin{align}
	\begin{split}
		\vec{x}_{i+1} = \vec{x}_{i+ \frac{1}{2}} + \frac{h}{2}\vec{v}_{i+\frac{1}{2}} \\
		\vec{v}_{i+1} = \vec{v}_{i+\frac{1}{2}} + \frac{h}{2}\vec{a}(\vec{x}_{i+1})
	\end{split} \label{velocity_verlet_split_2}
\end{align}

\begin{theorem} The Velocity Verlet algorithm preserves quadratic first integrals of the form $I(\vec{x}, \vec{v}) = \vec{v}^{T}\left(C\vec{x} + d\right)$ \end{theorem}
\begin{proof}
We have that $I'(\vec{x}, \vec{v}) = \vec{a}(\vec{x})^{T}\left(C\vec{x} + d\right) + \vec{v}^{T}C\vec{v} = 0$ for all $\vec{x}, \vec{v}$.
Now,
\begin{align*}
\vec{v}_{i+\frac{1}{2}}^{T}\left(C\vec{x}_{i+\frac{1}{2}} + d\right) &= \vec{v}_{i+\frac{1}{2}}^{T}\left(C\vec{x}_{i} + \frac{h}{2}C\vec{v}_{i+\frac{1}{2}} + d\right) \\
&= \vec{v}_{i+\frac{1}{2}}^{T}\left(C\vec{x}_{i} + d\right) + \frac{h}{2}\vec{v}_{i+\frac{1}{2}}^{T}C\vec{v}_{i+\frac{1}{2}} \\
&= \left(\vec{v}_{i} + \frac{h}{2}\vec{a}(\vec{x}_{i})\right)^{T}\left(C\vec{x}_{i} + d\right)+ \frac{h}{2}\vec{v}_{i+\frac{1}{2}}^{T}C\vec{v}_{i+\frac{1}{2}} \\
&=\vec{v}_{i}^{T}(C\vec{x}_{i} + d) + \frac{h}{2}\left( \vec{a}(\vec{x}_{i})^{T}(C\vec{x}_{i} + d) + \vec{v}_{i+\frac{1}{2}}^{T}C\vec{v}_{i+\frac{1}{2}}\right) \\
&=\vec{v}_{i}^{T}(C\vec{x}_{i} + d) + I'(\vec{x}_{i}, \vec{v}_{i+\frac{1}{2}}) \\
&=\vec{v}_{i}^{T}(C\vec{x}_{i} + d)
\end{align*}

Similarly, we get from Equation~\ref{velocity_verlet_split_2} that
\begin{align*}
I(\vec{x}_{i+1}, \vec{v}_{i+1}) &=\vec{v}_{i+1}^{T}(C\vec{x}_{i+1} + d) \\
&= \vec{v}_{i+\frac{1}{2}}^{T}\left(C\vec{x}_{i+\frac{1}{2}} + d\right) \\
&=\vec{v}_{i}^{T}(C\vec{x}_{i} + d) \\
&= I(\vec{x}_{i}, \vec{v}_{i})
\end{align*}
\end{proof}

It is unclear whether any other Newmark Beta method preserves angular momentum. No other Newmark Beta method can be expressed as the composition of two methods - so, we can't adapt the proof given above. This leads to a lot of unwieldly algebra.
We can test for parameters  numerically first, and then using the results to guide us towards the theoretical result. However, while running the simulation with the Velocity Verlet method, we get that the angular momentum is not preserved! Table~\ref{tbl:total_angular_momentum_50_iterations} shows the first five samples of total angular momentum taken during the simulation. 
\begin{table}[h]
	\centering
	\begin{tabular}{ |r|r|r|r| }
		\hline
		Iteration & x-component & y-component & z-component\\
		\hline
		10 & 493.28 & -402.32 & 1407.30 \\
		20 & -4.80 & -10.18 & 914.24 \\
		30 & 320.20 & -147.35 & 980.93 \\
		40 & 717.70 & -181.21 & 704.27 \\
		50 & 594.66 & 152.47 & 507.19 \\
		\hline
	\end{tabular}
	\caption{Total angular momentum of the system over the first 50 iterations}
	\label{tbl:total_angular_momentum_50_iterations}
\end{table}

The reason behind this incoherent behaviour is the application of periodic boundary conditions. Angular momentum is defined relative to the origin as it is the cross product of position and momentum. When a particle moves out of the box, it reenters the box from the opposite face due to periodic boundary conditions - this creates a discontinuity in the position of the particle, and hence, a discontinuity in the the angular momentum of the particle. Note however, that the velocity of the particle does not change, and hence, linear momentum is preserved. We say that periodic boundary conditions preserve translational symmetry, but not rotational symmetry.
As a result, we have that in the simulations, no method will preserve total angular momentum. However, this will not affect the simulation as no thermodynamic property depends upon the angular momenta of the particles.
 
\subsection{Total Energy}

\begin{theorem} The total energy, or the Hamiltonian, is a first integral. \end{theorem}
\begin{proof}
From definitions,
$$
	\nabla H = \left(\nabla_{p}H^{T}, \nabla_{q}H^{T}\right)
$$
and
$$
	\frac{dH}{dt} =
		\begin{pmatrix*}[r]
			-\nabla_{q}H \\
			\nabla_{p}H
		\end{pmatrix*}
$$
Then, 
$$
\nabla H(p, q) \cdot \frac{dH}{dt} = \nabla_{p}H^{T}\cdot-\nabla_{q}H + \nabla_{q}H^{T}\cdot \nabla_{p}H = 0
$$ 
\end{proof} 

The case of total energy is more complicated than those of the other geometric invariants. As we will see later, it is not the Hamiltonian is not preserved by some methods, but rather it is the `shadow' variant of the Hamiltonian.

We start by quoting Simo, Tarnow, and Wong from \cite{SimoTarnowWong1992}:
``\textit{What may seem surprising is that all of the implicit members of the Newmark family, perhaps the most widely used time-stepping algorithms in nonlinear structural dynamics, are not designed to conserve energy and also fail to conserve momentum. Among the explicit members, only the central difference method preserves momentum.}''
\end{document}
