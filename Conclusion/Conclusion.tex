\documentclass[../Main.tex]{subfiles}

\begin{document}

\section{Conclusion}

The Velocity Verlet algorithm clearly stands out from the rest of the algorithms. It is the only method that conserves all four properties of time reversibility, total linear momentum, total angular momentum, and total energy. Any other explicit Newmar Beta scheme conserves only linear momentum, a property common to the entire family of  Newmark Beta methods. As $\gamma \neq \frac{1}{2}$ introduces energy drift in the total energy of the system (proportional to $\gamma  - \frac{1}{2}$), we concentrate only on the family of Newmark Beta schemes with $\gamma = \frac{1}{2}$ and $\beta \in \left[0, \frac{1}{2}\right]$ - we will call this family the reversible Newmark Beta  methods, from Claim~\ref{clm:gamma_half_reversible}. However, we saw in practice that the schemes did not behave as expected on time reversibilty. While we cannot look at the mean pointwise absolute error in positions and velocities as a good metric (geometric numerical integration does not preserve individual trajectories - the velocity of a particle fluctuates throughout the simulation), we cannot ignore that the floating point arithmetic tends to include errors. Furthermore, as the value of force on a particle derived at the start of the simulation influences the behaviour of the particle for the rest of the simulation, the errors compound. Further focus on the time reversibility problem and ways to reduce this error are looked at in \cite{Levesque1993} - an interesting suggestion to not scale length by the value of $\sigma$, but rather by the interparticular distance in the initial cube lattice strucutre. This would then make each particle have an integer displacement, rather than a floating point one.
Only the explicit case of the reversible Newmark Beta methods preserves angular momentum - or quadratic first integrals of the form defined in Theorem~\ref{thm:specific_quadratic_first_integral_preservation}. However, this does not hold in practice, due to presence of periodic boundary conditions.
The case of total energy conservation is peculiar. While we can definitely say the Velocity Verlet algorithm conservers energy (to the order of $\mathcal{O}\left(h^{2}\right)$), we cannot make or deny this claim for any other reversible method due to a couple of reasons. Firstly, the computationally intensive nature of Newton-Raphson iteration on this scale prevents us from running longer trials to test the hypothesis. In the three cases that we did run for longer time frames, no distincitve difference in graphs or data was visible. We suspect that the Newton-Raphson iteration is not the best tool for computing the Newmark Beta algorithms. It is computation power and memory exhaustive - the $2592\times 2592$ Jacobian takes 53.7 MB of memory to be stored - and could be the reason behind the translated reversed positions of particles under the implicit reversible Newmark Beta methods.
Further steps from here would be to focus mainly on improving the runtime of the implicit Newmark Beta methods. As forces do not deviate by much between trials, we can generate a table of most common forces and use that to lookup the inter-particle force instead of computing them each time. We can do the same for the Jacobian, re-using it over a few trials to make computations faster. However, this could lead to added errors in the system. One way to make the program run faster would be to avoid using Newton-Raphson iterations completely, but no study has been done by us to compare how the predictor-corrector algorithms fare in comparison on the grounds of accuracy and speed of execution.
An interesting study would be to find the optimal initial configuration. We chose a lot of the parameters such as the cut-off distance and the update interval of neighbour lists on the basis of empirical evidence on a small sample, rather than an extensive exploration backed with mathematical fact. Understanding how varying the parameters influence the various results of the simulation would help identify the source of errors in the simulation.
\end{document}